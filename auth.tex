\documentclass{article}

\usepackage{url}

\RequirePackage[colorlinks]{hyperref}

\title{All Things Authentication}

\begin{document}

\maketitle

\section{Auth types}


\begin{enumerate}
  \item Basic Access Authentication
  \item OAuth2
  \item JWT
  \item SAML
\end{enumerate}

\section{Basic Auth}

From \href{https://en.wikipedia.org/wiki/Basic_access_authentication}{%
  Basic Access Authentication on Wikipedia}, is a method for an HTTP
user agent to provide a user and password when making a request.

The request contains a header field in the form \texttt{Authorization: Basic
 <credentials>}. Credentials are in the form \texttt{username:password} encoded in
Base64, that is, the \texttt{username} and \texttt{password} are separated by a colon
and Base64 encoded.

Some notes:
\begin{itemize}
  \item \href{https://tools.ietf.org/html/rfc7617}{%
      RFC 7617 The 'Basic' HTTP Authentication Scheme} is the
    canonical reference.
  \item MacOs has the \texttt{base64} available for the command line. Utilities
    exist in Ruby, Python, etc. for managing base64 encoding.
\end{itemize}

Some exercises:

\begin{itemize}
  \item Find some source code using Basic Auth and understand how
    it's implemented in that code.
  \item Examine the HTTP request headers for a service using Basic Auth
    to see how the header looks.
\end{itemize}

\section{OAuth2}


\section{JWT}

JSON Web Token (JWT) is a URL-safe means for representing
claims in JSON format to be transferred between two parties.

Some notes:

\begin{itemize}
  \item \href{https://en.wikipedia.org/wiki/JSON_Web_Token}{%
     JSON Web Token on Wikipedia}
  \item \href{https://tools.ietf.org/html/rfc7519}{%
     RFC 7519 JSON Web Token (JWT)}
 \item JWT may used to implement SSO.
\end{itemize}


\section{SAML}


\appendix

\section{Glossary}

\paragraph{base64url} \href{https://en.wikipedia.org/wiki/Base64#URL_applications}{%
  Base64Url} as described in \href{https://tools.ietf.org/html/rfc4648}{%
    RFC 4648 The Base16, Base32, and Base64 Data Encoding}

\section{Source code}

\end{document}

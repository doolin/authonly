\documentclass{article}

\usepackage{url}

\RequirePackage[colorlinks]{hyperref}

\title{All Things Authentication}

\begin{document}

\maketitle

\tableofcontents


\section{Basic Auth}

From \href{https://en.wikipedia.org/wiki/Basic_access_authentication}{%
  HTTP Basic Access Authentication on Wikipedia}, is a method for an HTTP
user agent to provide a user and password when making a request.

The request contains a header field in the form \texttt{Authorization: Basic
 <credentials>}. Credentials are in the form \texttt{username:password} encoded in
Base64, that is, the \texttt{username} and \texttt{password} are separated by a colon
and Base64 encoded.

Basic Auth (BA) is the simplest technique for enforcing access control to
web resources as it does not require cookies, session identifiers, or
login pages. API call authentication over HTTPS is a common use case.

Some notes:
\begin{itemize}
  \item \href{https://tools.ietf.org/html/rfc7617}{%
      RFC 7617 The 'Basic' HTTP Authentication Scheme} is the
    canonical reference.
  \item MacOs has the \texttt{base64} available for the command line. Utilities
    exist in Ruby, Python, etc. for managing base64 encoding.
  \item \href{https://api.rubyonrails.org/classes/ActionController/HttpAuthentication/Basic.html}{%
      Basic Auth is part of Rails}, hence available to every Rails application.
\end{itemize}

\subsection{Basic Auth in Rails}

The key is using \href{https://github.com/rails/rails/blob/master/actionpack/lib/action_controller/metal/http_authentication.rb}{%
Rails' built in Basic Auth system}.

The key is the \texttt{authenticate} metaprogramming
\href{https://github.com/rails/rails/blob/master/activemodel/lib/active_model/secure_password.rb#L119}{%
  defined in ActiveModel::SecurePassword}.

Here is \href{https://www.youtube.com/watch?v=O1sgFzn_Pgk}{%
a great video on macros in Ruby}; ruby macros are used extensively in Rails.

\subsection{Exercises}

\begin{itemize}
  \item Run the following from the command line: \texttt{%
      echo "username:password" | base64}. What is returned?
    I get \texttt{dXNlcm5hbWU6cGFzc3dvcmQK}.
  \item Examine the HTTP request headers for a service using Basic Auth
    to see how the header looks.
  \item Find some source code using Basic Auth and understand how
    it's implemented in that code. If you can't find anything, write
    a simple open source example for yourself.
\end{itemize}

\section{HTTP Digest}

HTTP Digest can be used when the client is a web browser. From Wikipedia,
digest authentication is an application of MD5 cryptographic hashing with usage
of nonce values to prevent replay attacks.

From \href{https://en.wikipedia.org/wiki/Digest_access_authentication}{HTTP Digest on Wikipedia}.

\begin{itemize}
  \item \href{https://tools.ietf.org/html/rfc7616}{%
      RFC 7616 HTTP Digest Access Authentication}
\end{itemize}

\section{HTTP Token}

\section{Form-based authentication}

\section{HMAC}

\section{OAuth}


\section{OAuth2}


\section{JWT}

JSON Web Token (JWT) is a URL-safe means for representing
claims in JSON format to be transferred between two parties.
Tokens are signed with either a private secret, or a public/private
key pair.

Some notes:

\begin{itemize}
  \item \href{https://en.wikipedia.org/wiki/JSON_Web_Token}{%
     JSON Web Token on Wikipedia}
  \item \href{https://tools.ietf.org/html/rfc7519}{%
     RFC 7519 JSON Web Token (JWT)}
 \item JWT may used to implement SSO.
 \item JWTs can be signed using a secret (with HMAC algorithm) or a
   public/private key pair using RSA.
 \item \href{https://auth0.com/learn/json-web-tokens/}{%
Get Started with JSON Web Tokens} from Auth0.
\end{itemize}


\section{SAML}

\section{Useful links}

\begin{itemize}
  \item \href{https://blog.restcase.com/restful-api-authentication-basics/}{%
      RESTful API Authentication Basics}
\end{itemize}


\section{OpenID}


\section{BrowserID}

\appendix

\section{Glossary}

\paragraph{base64url} \href{https://en.wikipedia.org/wiki/Base64#URL_applications}{%
  Base64Url} as described in \href{https://tools.ietf.org/html/rfc4648}{%
    RFC 4648 The Base16, Base32, and Base64 Data Encoding}

\section{Source code}

\section{TODO}

\begin{itemize}
  \item Write a simple, 1 file Rails application which demos Basic Auth
    using Rails built-in capability.
  \item Explain the difference in JWT between using a private secret and a
    public/private key pair.
  \item Read through source of \texttt{http\_authentication} to understand how
    the methods are implemented.
  \item Add HTTP Basic Auth to Anki deck, including all relevant material from
    RFC 7617.
  \item Can Rails Token Auth accept JWT?
\end{itemize}

\end{document}

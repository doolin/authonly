\section{Basic Auth}

From \href{https://en.wikipedia.org/wiki/Basic_access_authentication}{%
  HTTP Basic Access Authentication on Wikipedia}, is a method for an HTTP
user agent to provide a user and password when making a request. The request
contains a header field in the form \texttt{Authorization: Basic
 <credentials>}. Credentials are in the form \texttt{username:password} encoded in
Base64, that is, the \texttt{username} and \texttt{password} are separated by a colon
and Base64 encoded.

Basic Auth (BA) is the simplest technique for enforcing access control to
web resources as it does not require cookies, session identifiers, or
login pages. API call authentication over HTTPS is a common use case.

Some notes:
\begin{itemize}
  \item \href{https://tools.ietf.org/html/rfc7617}{%
      RFC 7617 The 'Basic' HTTP Authentication Scheme} is the
    canonical reference.
  \item MacOs has the \texttt{base64} available for the command line. Utilities
    exist in Ruby, Python, etc. for managing base64 encoding.
  \item \href{https://api.rubyonrails.org/classes/ActionController/HttpAuthentication/Basic.html}{%
      Basic Auth is part of Rails}, hence available to every Rails application.
\end{itemize}

\subsection{Basic Auth in Rails}

The key is using \href{https://github.com/rails/rails/blob/master/actionpack/lib/action_controller/metal/http_authentication.rb}{%
Rails' built in Basic Auth system}.

% TODO: Explain how the SecurePassword is used in a service.
The key is the \texttt{authenticate} metaprogramming
\href{https://github.com/rails/rails/blob/master/activemodel/lib/active_model/secure_password.rb#L119}{%
  defined in ActiveModel::SecurePassword}.

Here is \href{https://www.youtube.com/watch?v=O1sgFzn_Pgk}{%
a great video on macros in Ruby}; ruby macros are used extensively in Rails.

\subsection{Exercises}

\begin{itemize}
  \item Run the following from the command line: \texttt{%
      echo "username:password" | base64}. What is returned?
    I get \texttt{dXNlcm5hbWU6cGFzc3dvcmQK}.
  \item Examine the HTTP request headers for a service using Basic Auth
    to see how the header looks.
  \item Find some source code using Basic Auth and understand how
    it's implemented in that code. If you can't find anything, write
    a simple open source example for yourself.
\end{itemize}



\section{JWT}

JSON Web Token (JWT) is a URL-safe means for representing
claims in JSON format to be transferred between two parties.
Tokens are signed with either a private secret, or a public/private
key pair.

Some notes:

\begin{itemize}
  \item \href{https://en.wikipedia.org/wiki/JSON_Web_Token}{%
     JSON Web Token on Wikipedia}
  \item \href{https://tools.ietf.org/html/rfc7519}{%
     RFC 7519 JSON Web Token (JWT)}
 \item JWT may used to implement SSO.
 \item JWTs can be signed using a secret (with HMAC algorithm) or a
   public/private key pair using RSA.
 \item \href{https://auth0.com/learn/json-web-tokens/}{%
Get Started with JSON Web Tokens} from Auth0.
\end{itemize}

TODO: Search for how Rails can work with JWT, either as a ID server,
or as a client (or both?).


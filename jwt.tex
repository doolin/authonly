\section{JWT}

JSON Web Token (JWT) is a URL-safe means for representing
claims in JSON format to be transferred between two parties.
Tokens are signed with either a private secret, or a public/private
key pair.

% TODO: add a quad diagram of encryption and signing options
% TODO: create a quad diagram macro for general use

Some notes:

\begin{itemize}
  \item \href{https://en.wikipedia.org/wiki/JSON_Web_Token}{%
     JSON Web Token on Wikipedia}
  \item \href{https://tools.ietf.org/html/rfc7519}{%
     RFC 7519 JSON Web Token (JWT)}
 \item JWT may used to implement SSO.
 \item JWTs can be signed using a secret (with HMAC algorithm) or a
   public/private key pair using RSA.
 \item \href{https://auth0.com/learn/json-web-tokens/}{%
Get Started with JSON Web Tokens} from Auth0.
  \item In Ruby, the order of elements in a hash matters when signing, because the
        gets turned into a string.
\end{itemize}

TODO: Search for how Rails can work with JWT, either as a ID server,
or as a client (or both?).

TODO: Investigate ordering in simple Javascript objects corresponding to Ruby hashes.

TODO: Split the jwt example scripts into 3 parts.

\subsection{An example authentication server}

Three parts:

\begin{enumerate}
  \item Client 1: logs into Auth server, retrieves token, queries Service 1.
        Client 1 probably only needs to store whole token, without any processing of token.
  \item Service 1: accepts token from Client 1, returns data.
  \item Auth server: accepts login credentials from Client 1, returns token.
\end{enumerate}


% TODO: This picture needs to be turned into a quad which explains how JWE and JWS
% work within the JWT framework.
\begin{tikzpicture}[font=\sffamily]
  \SetEntries{feed marmot}{hibernate}{meet other marmots}{}{}{}{}{}
  \EisenBlock{mgreen}{tl}{Urgent and important}{Do first}
  \SetEntries{chat}{brush your teeth}{}{}{}{}{}{}
  \EisenBlock[right=1.2cm of multi-tl]{mblue}{tr}{Less urgent, but
  important}{Schedule}
  \SetEntries{clean up}{buy a calendar}{hold your breat}{}{}{}{}{}
  \EisenBlock[below=1.2cm of multi-tl]{orange}{bl}{Urgent, but
  less important}{Delegate}
  \SetEntries{eat onions}{}{}{}{}{}{}{}
  \EisenBlock[right=1.2cm of multi-bl]{red!80}{br}{Neither urgent nor
  important}{Don't do}
  \draw[ultra thick,-latex] (multi-tl.one east) to[out=0,in=180]
  (multi-tr.two west);
\end{tikzpicture}
\usepackage{url}
\usepackage{enumitem}

\RequirePackage[colorlinks]{hyperref}

%%% From https://tex.stackexchange.com/questions/447399/how-to-create-eisenhower-matrix-notepad-in-latex
\usepackage{tikz,amssymb}
\usetikzlibrary{shapes.multipart,positioning,fit,backgrounds}
\definecolor{mgreen}{RGB}{22,171,53}
\definecolor{mblue}{RGB}{22,101,171}

\newcommand{\EisenBlock}[5][]{
  \node [rectangle split,rectangle split parts=8,fill=white,
  text width=5cm,align=left,text=#2,draw,rounded corners,draw=#2,
  #1] 
  (multi-#3)
 {\strut$\Box$ \EntryOne\nodepart{two}\strut$\Box$ \EntryTwo
 \nodepart{three}\strut$\Box$ \EntryThree\nodepart{four}\strut$\Box$ \EntryFour
 \nodepart{five}\strut$\Box$ \EntryFive\nodepart{six}\strut$\Box$ \EntrySix
 \nodepart{seven}\strut$\Box$ \EntrySeven\nodepart{eight}\strut$\Box$ \EntryEight};
 \node[left=1pt of multi-#3.south west,anchor=south west,rotate=90,text=white] 
 (label-#3) {#4};
 \begin{scope}[on background layer]
 \node[fit=(multi-#3) (label-#3),fill=#2,rounded corners,
 label={[text=#2,anchor=south west,font=\bfseries]above left:#5}] (fit-#3){};
 \end{scope}
 \ClearEntries
}

\newcommand{\SetEntries}[8]{
\def\EntryOne{#1}
\def\EntryTwo{#2}
\def\EntryThree{#3}
\def\EntryFour{#4}
\def\EntryFive{#5}
\def\EntrySix{#6}
\def\EntrySeven{#7}
\def\EntryEight{#8}}
\newcommand{\ClearEntries}
{\SetEntries{\empty}{\empty}{\empty}{\empty}{\empty}{\empty}{\empty}{\empty}}
\ClearEntries
